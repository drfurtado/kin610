% Options for packages loaded elsewhere
\PassOptionsToPackage{unicode}{hyperref}
\PassOptionsToPackage{hyphens}{url}
\PassOptionsToPackage{dvipsnames,svgnames,x11names}{xcolor}
%
\documentclass[
  letterpaper,
  DIV=11,
  numbers=noendperiod]{scrartcl}

\usepackage{amsmath,amssymb}
\usepackage{lmodern}
\usepackage{iftex}
\ifPDFTeX
  \usepackage[T1]{fontenc}
  \usepackage[utf8]{inputenc}
  \usepackage{textcomp} % provide euro and other symbols
\else % if luatex or xetex
  \usepackage{unicode-math}
  \defaultfontfeatures{Scale=MatchLowercase}
  \defaultfontfeatures[\rmfamily]{Ligatures=TeX,Scale=1}
\fi
% Use upquote if available, for straight quotes in verbatim environments
\IfFileExists{upquote.sty}{\usepackage{upquote}}{}
\IfFileExists{microtype.sty}{% use microtype if available
  \usepackage[]{microtype}
  \UseMicrotypeSet[protrusion]{basicmath} % disable protrusion for tt fonts
}{}
\makeatletter
\@ifundefined{KOMAClassName}{% if non-KOMA class
  \IfFileExists{parskip.sty}{%
    \usepackage{parskip}
  }{% else
    \setlength{\parindent}{0pt}
    \setlength{\parskip}{6pt plus 2pt minus 1pt}}
}{% if KOMA class
  \KOMAoptions{parskip=half}}
\makeatother
\usepackage{xcolor}
\setlength{\emergencystretch}{3em} % prevent overfull lines
\setcounter{secnumdepth}{5}
% Make \paragraph and \subparagraph free-standing
\ifx\paragraph\undefined\else
  \let\oldparagraph\paragraph
  \renewcommand{\paragraph}[1]{\oldparagraph{#1}\mbox{}}
\fi
\ifx\subparagraph\undefined\else
  \let\oldsubparagraph\subparagraph
  \renewcommand{\subparagraph}[1]{\oldsubparagraph{#1}\mbox{}}
\fi


\providecommand{\tightlist}{%
  \setlength{\itemsep}{0pt}\setlength{\parskip}{0pt}}\usepackage{longtable,booktabs,array}
\usepackage{calc} % for calculating minipage widths
% Correct order of tables after \paragraph or \subparagraph
\usepackage{etoolbox}
\makeatletter
\patchcmd\longtable{\par}{\if@noskipsec\mbox{}\fi\par}{}{}
\makeatother
% Allow footnotes in longtable head/foot
\IfFileExists{footnotehyper.sty}{\usepackage{footnotehyper}}{\usepackage{footnote}}
\makesavenoteenv{longtable}
\usepackage{graphicx}
\makeatletter
\def\maxwidth{\ifdim\Gin@nat@width>\linewidth\linewidth\else\Gin@nat@width\fi}
\def\maxheight{\ifdim\Gin@nat@height>\textheight\textheight\else\Gin@nat@height\fi}
\makeatother
% Scale images if necessary, so that they will not overflow the page
% margins by default, and it is still possible to overwrite the defaults
% using explicit options in \includegraphics[width, height, ...]{}
\setkeys{Gin}{width=\maxwidth,height=\maxheight,keepaspectratio}
% Set default figure placement to htbp
\makeatletter
\def\fps@figure{htbp}
\makeatother
\newlength{\cslhangindent}
\setlength{\cslhangindent}{1.5em}
\newlength{\csllabelwidth}
\setlength{\csllabelwidth}{3em}
\newlength{\cslentryspacingunit} % times entry-spacing
\setlength{\cslentryspacingunit}{\parskip}
\newenvironment{CSLReferences}[2] % #1 hanging-ident, #2 entry spacing
 {% don't indent paragraphs
  \setlength{\parindent}{0pt}
  % turn on hanging indent if param 1 is 1
  \ifodd #1
  \let\oldpar\par
  \def\par{\hangindent=\cslhangindent\oldpar}
  \fi
  % set entry spacing
  \setlength{\parskip}{#2\cslentryspacingunit}
 }%
 {}
\usepackage{calc}
\newcommand{\CSLBlock}[1]{#1\hfill\break}
\newcommand{\CSLLeftMargin}[1]{\parbox[t]{\csllabelwidth}{#1}}
\newcommand{\CSLRightInline}[1]{\parbox[t]{\linewidth - \csllabelwidth}{#1}\break}
\newcommand{\CSLIndent}[1]{\hspace{\cslhangindent}#1}

\KOMAoption{captions}{tableheading}
\makeatletter
\@ifpackageloaded{tcolorbox}{}{\usepackage[many]{tcolorbox}}
\@ifpackageloaded{fontawesome5}{}{\usepackage{fontawesome5}}
\definecolor{quarto-callout-color}{HTML}{909090}
\definecolor{quarto-callout-note-color}{HTML}{0758E5}
\definecolor{quarto-callout-important-color}{HTML}{CC1914}
\definecolor{quarto-callout-warning-color}{HTML}{EB9113}
\definecolor{quarto-callout-tip-color}{HTML}{00A047}
\definecolor{quarto-callout-caution-color}{HTML}{FC5300}
\definecolor{quarto-callout-color-frame}{HTML}{acacac}
\definecolor{quarto-callout-note-color-frame}{HTML}{4582ec}
\definecolor{quarto-callout-important-color-frame}{HTML}{d9534f}
\definecolor{quarto-callout-warning-color-frame}{HTML}{f0ad4e}
\definecolor{quarto-callout-tip-color-frame}{HTML}{02b875}
\definecolor{quarto-callout-caution-color-frame}{HTML}{fd7e14}
\makeatother
\makeatletter
\makeatother
\makeatletter
\makeatother
\makeatletter
\@ifpackageloaded{caption}{}{\usepackage{caption}}
\AtBeginDocument{%
\ifdefined\contentsname
  \renewcommand*\contentsname{Table of contents}
\else
  \newcommand\contentsname{Table of contents}
\fi
\ifdefined\listfigurename
  \renewcommand*\listfigurename{List of Figures}
\else
  \newcommand\listfigurename{List of Figures}
\fi
\ifdefined\listtablename
  \renewcommand*\listtablename{List of Tables}
\else
  \newcommand\listtablename{List of Tables}
\fi
\ifdefined\figurename
  \renewcommand*\figurename{Figure}
\else
  \newcommand\figurename{Figure}
\fi
\ifdefined\tablename
  \renewcommand*\tablename{Table}
\else
  \newcommand\tablename{Table}
\fi
}
\@ifpackageloaded{float}{}{\usepackage{float}}
\floatstyle{ruled}
\@ifundefined{c@chapter}{\newfloat{codelisting}{h}{lop}}{\newfloat{codelisting}{h}{lop}[chapter]}
\floatname{codelisting}{Listing}
\newcommand*\listoflistings{\listof{codelisting}{List of Listings}}
\makeatother
\makeatletter
\@ifpackageloaded{caption}{}{\usepackage{caption}}
\@ifpackageloaded{subcaption}{}{\usepackage{subcaption}}
\makeatother
\makeatletter
\@ifpackageloaded{tcolorbox}{}{\usepackage[many]{tcolorbox}}
\makeatother
\makeatletter
\@ifundefined{shadecolor}{\definecolor{shadecolor}{rgb}{.97, .97, .97}}
\makeatother
\makeatletter
\makeatother
\ifLuaTeX
  \usepackage{selnolig}  % disable illegal ligatures
\fi
\IfFileExists{bookmark.sty}{\usepackage{bookmark}}{\usepackage{hyperref}}
\IfFileExists{xurl.sty}{\usepackage{xurl}}{} % add URL line breaks if available
\urlstyle{same} % disable monospaced font for URLs
% Make links footnotes instead of hotlinks:
\DeclareRobustCommand{\href}[2]{#2\footnote{\url{#1}}}
\hypersetup{
  pdftitle={KIN 610: Quantitative Analysis of Research in Kinesiology},
  colorlinks=true,
  linkcolor={blue},
  filecolor={Maroon},
  citecolor={Blue},
  urlcolor={Blue},
  pdfcreator={LaTeX via pandoc}}

\title{KIN 610: Quantitative Analysis of Research in Kinesiology}
\usepackage{etoolbox}
\makeatletter
\providecommand{\subtitle}[1]{% add subtitle to \maketitle
  \apptocmd{\@title}{\par {\large #1 \par}}{}{}
}
\makeatother
\subtitle{Department of Kinesiology, Cal State Northridge\\
Fall 2022 \textbar{} Tuesday, 4:00-6:45 p.m (RE 276)}
\author{}
\date{}

\begin{document}
\maketitle
\ifdefined\Shaded\renewenvironment{Shaded}{\begin{tcolorbox}[interior hidden, breakable, borderline west={3pt}{0pt}{shadecolor}, boxrule=0pt, frame hidden, enhanced, sharp corners]}{\end{tcolorbox}}\fi

\renewcommand*\contentsname{Table of contents}
{
\hypersetup{linkcolor=}
\setcounter{tocdepth}{3}
\tableofcontents
}
Download the syllabus as \href{syllabus.pdf}{PDF}

\hypertarget{sec-instructor-info}{%
\section{Instructor Info}\label{sec-instructor-info}}

Ovande Furtado Jr., Ph.D.

Dr.~Furtado received a B.A. in Physical Education from the Federal
University of Parana, Curitiba, PR - Brazil. He earned his M.S. and
Ph.D.~degrees in Motor Behavior from the University of Pittsburgh, PA.
Dr.~Furtado's line of research focuses on two main areas: (1) validation
of observational models in psychomotor assessment instruments and (2)
the relationship between motor skill competence, perceived motor
competence, physical activity levels, and body composition.

\textbf{Office Hours}

See Section~\ref{sec-office-hours} for more information

\textbf{Contact Info}

Email: ovandef@csun.edu\\
Office: Redwood Hall 289

\hypertarget{sec-course-description}{%
\subsection{Course Description}\label{sec-course-description}}

This course focuses on the introductory statistical techniques used in
social science research. Students will be introduced to concepts such as
reliability, validity, measures of central tendency, variability,
probability, and statistical techniques including: t tests (independent
\& dependent samples), Analysis of Variance (ANOVA), Chi-square,
correlation, and regression.

Students are expected to take the material/concepts presented in this
course and apply them through a series of homework assignments and
quizzes. The overall goal of the course is not only to help students
understand the mathematical/statistical concepts presented but also to
assist in the application of these procedures.

\hypertarget{expectations-and-goals}{%
\subsection{Expectations and Goals}\label{expectations-and-goals}}

Upon completion of this course, you will be able to adequately:

\begin{enumerate}
\def\labelenumi{\arabic{enumi}.}
\tightlist
\item
  Introduce statistical concepts utilized in research within the social
  sciences
\item
  Apply the mathematical/statistical techniques presented for social
  science research
\item
  Demonstrate an ability to analyze and interpret data within the social
  sciences
\item
  Provide practical examples as to when statistical techniques presented
  are appropriate methods for analysis.
\end{enumerate}

\hypertarget{text-readings-instructional-resources}{%
\subsection{Text, readings \& instructional
resources}\label{text-readings-instructional-resources}}

\hypertarget{required-ebook-free}{%
\subsubsection{Required eBook (free)}\label{required-ebook-free}}

Navarro and Foxcroft (2022)

\hypertarget{optional-textbook}{%
\subsubsection{Optional Textbook}\label{optional-textbook}}

Weir and Vincent (2021)

\hypertarget{other-readings}{%
\subsubsection{Other readings}\label{other-readings}}

All extra study content for this course can be found in the
\href{https://drfurtado.github.io/kin610/}{course's website}.

\hypertarget{instructional-resources}{%
\subsubsection{Instructional resources}\label{instructional-resources}}

\begin{itemize}
\tightlist
\item
  jamovi Statistical Software (The jamovi project 2021) (Free)
\item
  jamovi Video Tutorials Poulson (2019) (Free)
\end{itemize}

\hypertarget{sec-structure-requirements}{%
\subsection{Structure \&
Requirements}\label{sec-structure-requirements}}

I will adopt the 4 ``Ps''\footnote{The schedule is subject to change.}
in this course. This means that while taking this course you will be
asked to \texttt{prepare}, \texttt{participate}, \texttt{practice}, and
\texttt{perform}.

You are responsible for the material covered in class prior to attending
each class. Note that the week's readings are specified in the
\href{index.qmd}{course schedule}.

In addition to these readings, the instructor may assign supplemental
readings throughout the semester. These supplemental readings do not
appear on the schedule as these readings will be assigned at the
instructor's discretion.

The assignments used to enhance your learning experience in this course
include:

\hypertarget{sec-participation-attendance}{%
\subsubsection{Participation \&
Attendance}\label{sec-participation-attendance}}

Class presence and participation points are given to encourage your
active class participation and discussion. You will be rewarded with a
perfect score
\texttt{as\ long\ as\ you\ frequently\ come\ to\ class\ and\ actively\ contribute\ to\ the\ class\ discussion\ during\ lecture}s.

Although it is not required, most students send their professor a brief
e-mail to explain their absence in advance.

\hypertarget{preparedness}{%
\subsubsection{Preparedness}\label{preparedness}}

You will be evaluated on your preparedness by completing the a
\texttt{quiz} and the \texttt{major\ takeaways} assignment
\texttt{before\ each\ class}.

\hypertarget{sec-quizzes}{%
\subsubsection{Quizzes}\label{sec-quizzes}}

\texttt{Before} each class, you will complete a multiple-choice quiz on
the week's topic. You must score 100\% on each quiz. If you score below
100\%, you will have to retake the quiz until you score 100\%. You can
only move to the following quiz if you score 100\% on a quiz.

\hypertarget{labs}{%
\subsubsection{Labs}\label{labs}}

Students will complete several labs in this course. The purpose of each
lab is to assist students in applying their understanding of the
statistical procedures discussed in class as well as to provide an
opportunity for students to respond to the readings.

\begin{tcolorbox}[enhanced jigsaw, toptitle=1mm, left=2mm, colframe=quarto-callout-warning-color-frame, titlerule=0mm, opacityback=0, colbacktitle=quarto-callout-warning-color!10!white, opacitybacktitle=0.6, leftrule=.75mm, toprule=.15mm, bottomrule=.15mm, rightrule=.15mm, bottomtitle=1mm, breakable, title=\textcolor{quarto-callout-warning-color}{\faExclamationTriangle}\hspace{0.5em}{Warning}, coltitle=black, arc=.35mm, colback=white]

You are allowed to discuss the labs with other students (and with the
instructor), but you must write the final answers yourself in your own
words. Solutions prepared ``in committee'' or by, copying or
paraphrasing someone else's work is not acceptable; your hand-in
assignments must represent your thoughts.

\end{tcolorbox}

\hypertarget{sec-exams}{%
\subsubsection{Exams}\label{sec-exams}}

You will complete two (2) exams in this course. Students may use their
notes and textbook for the exams, but no outside resource other than a
calculator can be used.

Each exam has between six to ten questions, with each question worth 10
points. Exams must be completed in the allotted time. The exams (and
quizzes) focus on concepts and interpretation, with most of the
computational activities occurring in the homework assignments.

Although the quizzes and exams will not focus on previously tested
material (they are not meant to be cumulative), knowledge of previously
tested material may be inherently required to answer questions related
to new material.

Most of the computational activities will be via lab assignments. In
addition, selected readings will be assigned throughout the semester.
The content of these readings will be included in exam and quiz
questions and homework assignments.

\hypertarget{sec-duxe9juxe0-vu}{%
\subsubsection{\texorpdfstring{\emph{Déjà
vu}}{Déjà vu}}\label{sec-duxe9juxe0-vu}}

Organization is a prerequisite for effective learning. Throughout the
semester, you will be asked to organize the material presented in class
in a single directory in Google Drive.

The directory should have a Doc file (essential links), the syllabus in
PDF format, and several subdirectories for each topic covered in the
course. Inside each subdirectory, you must include the week's lesson in
PDF format and any assignments or activities you did. The structure of
the main directory should look like this:

KIN610

\begin{itemize}
\tightlist
\item
  Essential Links
\item
  Syllabus in pdf
\item
  Navarro and Foxcroft (2022) ebook
\end{itemize}

Week 1: \textless topic\textgreater{}

\begin{itemize}
\tightlist
\item
  Lesson in pdf
\item
  Any assignment and/or activity completed
\end{itemize}

Week 2 \textless topic\textgreater{} \ldots.

\hypertarget{sec-course-policy}{%
\subsection{Course Policy}\label{sec-course-policy}}

I will detail the policy for this course below. Basically, don't cheat
and try to learn stuff.

\hypertarget{grading}{%
\subsubsection{Grading}\label{grading}}

\begin{longtable}[]{@{}ll@{}}
\toprule()
Assignment & Percentage \\
\midrule()
\endhead
Participation \& Attendance & 5\% \\
Weekly Quizzes & 10\% \\
Major Takeaways\footnote{Class Dates: Aug 29, 2022 - Dec 12, 2022} &
20\% \\
Labs & 20\% \\
Exam 1 & 20\% \\
Exam 2 & 20\% \\
\emph{Déjà vu} & 5\% \\
\bottomrule()
\end{longtable}

\hypertarget{grading-scale}{%
\subsubsection{Grading Scale}\label{grading-scale}}

A 93.00-100.00 \textbar{} A- 90.00-92.99 B+ 87.00-89.99 \textbar{} B
83.00-86.99 \textbar{} B- 80.00-82.99 C+ 77.00-79.99 \textbar{} C
73.00-76.99 \textbar{} C- 70.00-72.99 D+ 67.00-69.99 \textbar{} D
63.00-66.99 \textbar{} D- 60.00-62.99 F \textless59.99

\begin{tcolorbox}[enhanced jigsaw, toptitle=1mm, left=2mm, colframe=quarto-callout-note-color-frame, titlerule=0mm, opacityback=0, colbacktitle=quarto-callout-note-color!10!white, opacitybacktitle=0.6, leftrule=.75mm, toprule=.15mm, bottomrule=.15mm, rightrule=.15mm, bottomtitle=1mm, breakable, title=\textcolor{quarto-callout-note-color}{\faInfo}\hspace{0.5em}{Note}, coltitle=black, arc=.35mm, colback=white]

In recognition of the fact that grading, however carefully done, will
always be imperfect, this class will utilize a ``round up'' rule for
assigning final grades. I will round up from .5\% and above, but
anything below this will round down. In other words, 79.5 will round up
to 80, while 79.4 will round down to 79 even.

\end{tcolorbox}

\begin{tcolorbox}[enhanced jigsaw, toptitle=1mm, left=2mm, colframe=quarto-callout-important-color-frame, titlerule=0mm, opacityback=0, colbacktitle=quarto-callout-important-color!10!white, opacitybacktitle=0.6, leftrule=.75mm, toprule=.15mm, bottomrule=.15mm, rightrule=.15mm, bottomtitle=1mm, breakable, title=\textcolor{quarto-callout-important-color}{\faExclamation}\hspace{0.5em}{Important}, coltitle=black, arc=.35mm, colback=white]

Requests for an Incomplete (I) must conform to
\href{https://bit.ly/3bDxwZi}{university policies}. Among other
requirements, ``I'' is possible only for instances in which you are
demonstrating passing work in the class.

\end{tcolorbox}

\hypertarget{attendance}{%
\subsubsection{Attendance}\label{attendance}}

\begin{quote}
\emph{Showing up is 80 percent of life} -- Woody Allen,
\href{http://quoteinvestigator.com/2013/06/10/showing-up/\#note-6553-1}{via
Marshall Brickman}
\end{quote}

Attendance will be taken at the beginning of every class; please, plan
accordingly.

\hypertarget{sec-e-mail}{%
\subsubsection{E-mail}\label{sec-e-mail}}

Please, do not use the built-in email (Inbox) in Canvas. Instead, use
your CSUN Gmail to communicate with me.

If your message concerns a non-private matter (e.g., assignments,
content, deadlines, etc.), then please post your question to our mailing
list\textbf{, which can be answered by any student taking the course.}
The mailing list address is provided in Canvas.

\hypertarget{sec-office-hours}{%
\subsubsection{Office Hours}\label{sec-office-hours}}

\hypertarget{in-person}{%
\paragraph{In-person}\label{in-person}}

Thursdays from 2-4 pm at RE 289.

\hypertarget{online-via-zoom}{%
\paragraph{Online via Zoom}\label{online-via-zoom}}

By appointment only:
\href{https://calendly.com/drfurtado}{www.calendly.com/drfurtado}

\hypertarget{sec-late-assignments}{%
\subsubsection{Late Assignments}\label{sec-late-assignments}}

It is important to note that late assignments are assessed a 10\%
deduction for each day it is late, not to exceed four days. After the
fourth day of the deadline, no assignments will be accepted. Therefore,
it is important to plan ahead and submit all assignments on time to
receive full credit for your work. The instructor reserves the right to
make exceptions to this policy on a case-by-case basis.

\hypertarget{extra-credit}{%
\subsubsection{Extra Credit}\label{extra-credit}}

There is no individual extra credit granted. Therefore, do not plan to
make-up poor grades at the end of the semester by asking to do extra
credit work. I might provide extra credit opportunities, but these will
be offered to the entire class, not to individuals.

\hypertarget{sec-disabilities}{%
\subsubsection{Disabilities}\label{sec-disabilities}}

Federal law mandates the provision of services at the university-level
to qualified students with disabilities.

This instructor, in conjunction with California State University
Northridge, is committed to upholding and maintaining all aspects of the
federal Americans with Disabilities Act of 1990 (ADA) and Section 504 of
the Rehabilitation Act of 1973.

If you are a student with a disability and wish to request
accommodations, please contact the office of Students with Disabilities
Resources located in 110 Student Services Building, or call (818)
677-2684 for an appointment. Any information regarding your disability
will remain confidential. Because many accommodations require early
planning, requests for accommodations should be made as early as
possible. Any requests for accommodations will be reviewed in a timely
manner to determine their appropriateness to this setting.

\hypertarget{dishonesty}{%
\subsubsection{Academic Dishonesty}\label{dishonesty}}

TL;DR: Don't cheat!

\emph{Please, \textbf{stop} and read the information below; this is
important!}

\begin{tcolorbox}[enhanced jigsaw, toptitle=1mm, left=2mm, colframe=quarto-callout-important-color-frame, titlerule=0mm, opacityback=0, colbacktitle=quarto-callout-important-color!10!white, opacitybacktitle=0.6, leftrule=.75mm, toprule=.15mm, bottomrule=.15mm, rightrule=.15mm, bottomtitle=1mm, breakable, title=\textcolor{quarto-callout-important-color}{\faExclamation}\hspace{0.5em}{Important}, coltitle=black, arc=.35mm, colback=white]

Each student is expected to be familiar with, and abide by, the
conditions of student conduct, as presented in the CSUN Catalog, with
emphasis on sections entitled, Student Conduct Code, Academic
Dishonesty, Faculty Policy on Academic Dishonesty, and Penalties. Any
student engaging in academic dishonesty (e.g., cheating, fabrication,
facilitating academic dishonesty, plagiarism) is subject to discipline,
which may include a failing grade in the course, and may also be subject
to more severe discipline by the University. Students are encouraged to
visit the link below and become familiar with the Standards for Student
Conduct.

\url{http://www.csun.edu/a\&r/soc/studentconduct.html}

\end{tcolorbox}

\hypertarget{about-plagiarism}{%
\paragraph{About Plagiarism}\label{about-plagiarism}}

Plagiarism means using words, ideas, or arguments from another person or
source without citation\footnote{Students are expected to read and study
  the assigned chapters prior to attending each class meeting.}. Cite
all sources consulted to any extent (including material from the
internet), whether or not assigned and whether or not quoted directly.
For quotations, four or more words used in sequence must be set off in
quotation marks, with the source identified.

Plagiarism is a serious violation of the CSUN Student Conduct Code.. Any
form of cheating will immediately earn you a failing grade for the
entire course. \textbf{By remaining enrolled, you consent to this
policy}.

\begin{quote}
Turnitin (see below) will detect such misconducts as it checks every
submission against a database of papers, as well as against the
Internet.
\end{quote}

What is Turnitin?

\href{https://www.turnitin.com/}{Turnitin} is an automated system that
instructors can use to quickly and easily compare each student's
assignment with billions of websites, as well as an enormous database of
student papers that grows with each submission. Accordingly, you will be
expected to submit assignments through the Canvas Assignment Tool in
electronic format. After the assignment is processed, as an instructor,
I receive a report from \emph{Turnitin} that states if and how another
author's work was used in the assignment.

\hypertarget{sec-final-notes}{%
\subsection{Final (yet important) Notes}\label{sec-final-notes}}

\hypertarget{how-to-access-our-course-and-get-started}{%
\subsubsection{How to Access our Course and Get
Started}\label{how-to-access-our-course-and-get-started}}

\begin{itemize}
\tightlist
\item
  Log into Canvas: \url{https://canvas.csun.edu}
\item
  Under ``My Courses,'' locate our course and click on it.
\item
  This will take you to the course home page.
\end{itemize}

\hypertarget{technology-requirements-and-support}{%
\subsubsection{Technology Requirements and
Support:}\label{technology-requirements-and-support}}

\begin{itemize}
\tightlist
\item
  A computer and access to the internet (reliable connection)
\item
  Firefox, Safari, etc. (web browser)
\end{itemize}

\hypertarget{what-i-expect-of-you}{%
\subsubsection{What I Expect of You:}\label{what-i-expect-of-you}}

\begin{enumerate}
\def\labelenumi{\arabic{enumi}.}
\tightlist
\item
  Plan your schedule to ensure you several hours per week to spend on
  this class and take time to identify where and when you'll do your
  learning.
\item
  Review the due dates for the assignments (refer to our Course Schedule
  in Canvas) to orient yourself to the flow of the learning.
\item
  This course requires regular engagement and practice using
  \protect\hyperlink{instructional-resources}{jamovi} (Statistical
  Package).
\end{enumerate}

\hypertarget{sec-success}{%
\subsection{How to be Success in this Course}\label{sec-success}}

Consider the goals you have for engaging in this course as you determine
how to allocate time to complete course requirements.

Each student has a different pace when comes to studying for a course.
Thus, I will let you figure out how many hours you need to reserve each
week for this course. Regardless of the number of hours chosen, try to
divide your time so that you devote more time to \texttt{assignments}
and \texttt{assigned\ readings}.

\hypertarget{sec-student-support-services}{%
\subsection{Student Support
Services}\label{sec-student-support-services}}

CSUN aims to make all learning experiences as accessible as possible,
and has a variety of resources available to help support students. If
you believe the design of this course poses barriers to effectively
participate or demonstrate your learning, please contact me to discuss
possible options and adjustments.

\begin{itemize}
\tightlist
\item
  The IT Help Center (818)677-1400, helpcenter@csun.edu is available to
  help with Canvas, CSUN e-mail, SOLAR/Portal, and other technical
  issues.
\item
  CSUN Device Loaner Program (\url{https://bit.ly/3t1G0An}) provides
  devices that can be checked out that includes laptops, webcams,
  hotspots and headsets
\item
  The Learning Resource Center (818) 677-2033 The mission of the LRC is
  to enable students to improve their academic performance through a
  variety of learning programs, including workshops, one-on-one and
  group tutoring, supplemental instruction classes and interactive
  subject area computer programs and videos. Student who use the LRC
  learning programs will develop and strengthen their critical thinking
  skills, study strategies, writing skills and performance in subject
  matter courses.
\item
  University Counseling Services (818) 677-2366, Bayramian Hall 520. UCS
  provides resources and information to assist students in dealing with
  a variety of large and small psychological obstacles that may
  interfere with academic progress and/or relationship satisfaction.
  Services include individual, group, and crisis counseling.
\item
  In accordance with the CSUN Accessibility Policy
  (\url{https://bit.ly/3yqGHE9}), CSUN is working to ensure that campus
  communication and course materials are accessible to everyone. Please
  reach out to me if you have difficulty with any of the materials for
  this course.
\item
  If you have a disability and need accommodations, please register with
  the Disability Resources and Educational Services (DRES) office or the
  National Center on Deafness (NCOD).

  \begin{itemize}
  \tightlist
  \item
    The DRES office can be reached at (818) 677-2684.
  \item
    NCOD can be reached at (818) 677-2611.
  \item
    Reasonable accommodations and services will be provided to students
    if requests are made in a timely manner and with appropriate
    documentation
  \item
    If you would like to discuss your need for accommodations with me,
    please drop in office hours or contact me to set up an appointment.
  \end{itemize}
\item
  Food Pantry (\url{https://bit.ly/38nTsVH}) at CSUN: Anybody who faces
  challenges securing food or housing and believes this impacts course
  performance, should contact CSUN's Food Pantry website and the
  corresponding contacts. If you also feel comfortable contacting me,
  the department chair, or the Dean's Office, we can also facilitate
  assistance. You don't have to be alone in this moment.
\item
  Emergency MataCare grants (\url{https://bit.ly/2WAZkIz}), one-time
  grants to prevent evictions, urgent child care issues, etc. - DACA
  (Deferred Action for Childhood Arrivals) Resources: Check out the
  Central American Resource Center facebook page
  (https://bit.ly/2Yg0p9z), legal resources listed on CSUN's Educational
  Opportunity Program (EOP) Dream Center that was created to support all
  undocumented students \& allies (Dream Center flyer). CSUN President
  Harrison issued a support statement on the CSUN homepage for DACA and
  resources.
\item
  Help lines (https://bit.ly/3sYbMOo)(after hours when the University
  Counseling is closed) for numerous topics/needs (e.g., suicide, drug,
  rape, LGBQT, military, or any crisis). You don't have to manage these
  feelings alone.
\item
  Pride Center (https://bit.ly/3jqNZUi) offers support and resources to
  lesbian, gay, bisexual, transgender, queer, \& questioning students,
  faculty, \& staff.
\item
  Klotz Student Health Center (https://bit.ly/3zx1Y0s): Numerous health
  services including primary care, dental, nutritional counseling,
  acupuncture, massage and lots more.
\item
  Career Center (https://bit.ly/3jtTcL2) for resume writing \&
  interviewing and much more; Matty's Closet (https://bit.ly/3jAResx)
  has free professional clothes for students who need interview or
  professional attire.
\item
  USU 9https://bit.ly/38uz59j) for more student services; Clubs \&
  Organizations (https://bit.ly/38tBhOa): Hopefully a dozen people have
  already advised you to ``get involved'' (https://bit.ly/3ysqYVb) at
  CSUN in something that interests you.
\item
  Associated Students (https://bit.ly/3yuWjGT) offers recycling, and a
  Children's Center providing child care
\item
  Financial Aid \& Scholarships (https://bit.ly/3sYFzqr) offers aid for
  applications
\item
  University Library https://bit.ly/3yuIEQ9) for many additional
  academic resources
\item
  Veterans Resource Center (https://bit.ly/38qYtg7) assists CSUN
  students as they transition from military service to academic success.
\end{itemize}

Title 5, California Code of Regulations,§ 41301. Standards for Student
Conduct -- (a) Campus Community Values: The university is committed to
maintaining a safe and healthy living and learning environment for
students, faculty, and staff. Each member of the campus community should
choose behaviors that contribute toward this end. Students are expected
to be good citizens and to engage in responsible behaviors that reflect
well upon their university, to be civil to one another and to others in
the campus community, and contribute positively to student and
university life.

CSUN with A HEART If you are facing challenges related to food
insecurity, housing precarity/homelessness, mental health, access to
technology, eldercare/childcare, or healthcare, you can find guidance,
help, and resources from CSUN with A HEART (https://www.csun.edu/heart).

\hypertarget{course-schedule}{%
\subsection{Course Schedule}\label{course-schedule}}

Optional Textbook: Weir and Vincent (2021)

\begin{longtable}[]{@{}
  >{\raggedright\arraybackslash}p{(\columnwidth - 8\tabcolsep) * \real{0.1158}}
  >{\raggedright\arraybackslash}p{(\columnwidth - 8\tabcolsep) * \real{0.1526}}
  >{\raggedright\arraybackslash}p{(\columnwidth - 8\tabcolsep) * \real{0.1000}}
  >{\raggedright\arraybackslash}p{(\columnwidth - 8\tabcolsep) * \real{0.3368}}
  >{\raggedright\arraybackslash}p{(\columnwidth - 8\tabcolsep) * \real{0.2789}}@{}}
\toprule()
\begin{minipage}[b]{\linewidth}\raggedright
\end{minipage} & \begin{minipage}[b]{\linewidth}\raggedright
Week\footnote{The schedule is subject to change.}
\end{minipage} & \begin{minipage}[b]{\linewidth}\raggedright
Date\footnote{Class Dates: Aug 29, 2022 - Dec 12, 2022}
\end{minipage} & \begin{minipage}[b]{\linewidth}\raggedright
Reading\footnote{Students are expected to read and study the assigned
  chapters prior to attending each class meeting.}
\end{minipage} & \begin{minipage}[b]{\linewidth}\raggedright
Assignments\footnote{Quizzes and Major Takeaways assignments are due
  \texttt{before} class; other assignment are due after class.}
\end{minipage} \\
\midrule()
\endhead
\begin{minipage}[t]{\linewidth}\raggedright
\end{minipage} & \href{/weeks/week-1.qmd}{Wk01} & \texttt{Jan\ 26} &
Course Intro \& Data Collection & Read and study the
\href{course-syllabus.qmd}{Syllabus} \\
\begin{minipage}[t]{\linewidth}\raggedright
\end{minipage} & \href{/weeks/week-2.qmd}{WK02} & \texttt{Feb\ 2} &
Introduction to statistical packages - jamovi and SPSS &
\texttt{Lab\ 1} \\
\begin{minipage}[t]{\linewidth}\raggedright
\end{minipage} & \href{/weeks/week-3.qmd}{WK03} & \texttt{Feb\ 9} &
Introduction to Statistics and Measurement & Quiz; Major Takeaways \\
\begin{minipage}[t]{\linewidth}\raggedright
\end{minipage} & \href{/weeks/week-4.qmd}{WK04} & \texttt{Feb\ 16} &
\begin{minipage}[t]{\linewidth}\raggedright
Organizing and Displaying Data\\
Percentiles\strut
\end{minipage} & Quiz; Major Takeaways \\
\begin{minipage}[t]{\linewidth}\raggedright
\end{minipage} & \href{/weeks/week-5.qmd}{WK05} & \texttt{Feb\ 23} &
\begin{minipage}[t]{\linewidth}\raggedright
Measures of Central Tendency\\
Measures of Variability\strut
\end{minipage} & Quiz; Major Takeaways; \texttt{Lab\ 2} \\
\begin{minipage}[t]{\linewidth}\raggedright
\end{minipage} & \href{/weeks/week-6.qmd}{WK06} & \texttt{Mar\ 2} &
\begin{minipage}[t]{\linewidth}\raggedright
Fundamentals of Inferential Statistical\\
\strut
\end{minipage} & Quiz; Major Takeaways \\
\begin{minipage}[t]{\linewidth}\raggedright
\end{minipage} & \href{/weeks/week-7.qmd}{WK07} & \texttt{Mar\ 9} &
Correlation and Bivariate Regression & Quiz; Major Takeaways \\
\begin{minipage}[t]{\linewidth}\raggedright
\end{minipage} & \href{/weeks/week-8.qmd}{WK08} & \texttt{Mar\ 16} &
Multiple Correlation and Multiple Regression & Quiz; Major Takeaways;
\texttt{Lab\ 3} \\
\begin{minipage}[t]{\linewidth}\raggedright
\end{minipage} & \href{/weeks/week-9.qmd}{WK09} & \texttt{Mar\ 23} &
\texttt{Exam\ 1}\footnote{Covers all previously covered topics.} & na \\
\begin{minipage}[t]{\linewidth}\raggedright
\end{minipage} & \href{/weeks/week-10.qmd}{WK10} & \texttt{Mar\ 30} &
The Student's \emph{t}-test & Quiz; Major Takeaways \\
\begin{minipage}[t]{\linewidth}\raggedright
\end{minipage} & \href{/weeks/week-11.qmd}{WK11} & \texttt{Apr\ 6} &
One-way Analysis of Variance & Quiz; Major Takeaways \\
\begin{minipage}[t]{\linewidth}\raggedright
\end{minipage} & \href{/weeks/week-12.qmd}{WK12} & \texttt{Apr\ 13} &
Analysis of Variance With Repeated Measures & Quiz; Major Takeaways;
\texttt{Lab\ 4} \\
\begin{minipage}[t]{\linewidth}\raggedright
\end{minipage} & \href{/weeks/week-13.qmd}{WK13} & \texttt{Apr\ 20} &
\begin{minipage}[t]{\linewidth}\raggedright
\hfill\break
Factorial Analysis of Variance: Between-Between\strut
\end{minipage} & Quiz; Major Takeaways \\
\begin{minipage}[t]{\linewidth}\raggedright
\end{minipage} & \href{/weeks/week-14.qmd}{WK14} & \texttt{Apr\ 27} &
\begin{minipage}[t]{\linewidth}\raggedright
\hfill\break
Factorial Analysis of Variance: Between-Within, Within-Within\strut
\end{minipage} & Quiz; Major Takeaways \\
\begin{minipage}[t]{\linewidth}\raggedright
\end{minipage} & \href{/weeks/week-15.qmd}{WK15} & \texttt{May\ 4} &
Reliability & Quiz; Major Takeaways; \texttt{Lab\ 5} \\
\begin{minipage}[t]{\linewidth}\raggedright
\end{minipage} & \href{/weeks/week-16.qmd}{WK16} & \texttt{May\ 11} &
Analysis of Nonparametric Data & Quiz; Major Takeaways \\
\begin{minipage}[t]{\linewidth}\raggedright
\end{minipage} & Final's Week &
\begin{minipage}[t]{\linewidth}\raggedright
\texttt{May\ 18}\strut \\
5:30PM - 7:30PM\\
Redwood Hall 276\strut
\end{minipage} & \texttt{Exam\ 2}\footnote{Covers mainly topics
  presented after Exam 1.} & na \\
\bottomrule()
\end{longtable}

\hypertarget{references}{%
\subsection*{References}\label{references}}
\addcontentsline{toc}{subsection}{References}

\hypertarget{refs}{}
\begin{CSLReferences}{1}{0}
\leavevmode\vadjust pre{\hypertarget{ref-navarro2022}{}}%
Navarro, Danielle J, and David R Foxcroft. 2022. \emph{Learning
Statistics with Jamovi: A Tutorial for Psychology Students and Other
Beginners (Version 0.75)}. Danielle J. Navarro; David R. Foxcroft.
\url{https://doi.org/10.24384/HGC3-7P15}.

\leavevmode\vadjust pre{\hypertarget{ref-poulson2019}{}}%
Poulson, Barton. 2019. {``Jamovi Video Tutorials.''}
\url{https://www.youtube.com/playlist?list=PLkk92zzyru5OAtc_ItUubaSSq6S_TGfRn}.

\leavevmode\vadjust pre{\hypertarget{ref-thejamoviproject2021}{}}%
The jamovi project. 2021. \emph{Jamovi}. Syndey, Australia.
\href{Retrieved\%20from\%20https://www.jamovi.org}{Retrieved from
https://www.jamovi.org}.

\leavevmode\vadjust pre{\hypertarget{ref-weir2021}{}}%
Weir, Joseph P., and William J. Vincent. 2021. \emph{Statistics in
Kinesiology}. Human Kinetics.
\url{https://us.humankinetics.com/products/statistics-in-kinesiology-5th-edition-with-web-resource}.

\end{CSLReferences}



\end{document}
